\documentclass[tikz, border=10pt]{standalone}
\usepackage{fontspec}
\usepackage{tikz}
\usepackage{xcolor}

% Load TikZ libraries for positioning, shapes, arrows, and decorations
\usetikzlibrary{shapes.geometric, positioning, arrows.meta, decorations.pathreplacing, calc}

% Define custom colors for the gradients and background
\definecolor{colorConvLeft}{RGB}{218,218,255}    % Light purple
\definecolor{colorConvRight}{RGB}{204,238,255}   % Light blue
\definecolor{colorBottleLeft}{RGB}{255,229,180}  % Light yellow
\definecolor{colorBottleRight}{RGB}{255,217,217} % Light pink
\definecolor{colorConcatLeft}{RGB}{179,229,252}  % Light cyan
\definecolor{colorConcatRight}{RGB}{255,170,170} % Light red
\definecolor{bordercolor}{RGB}{68,84,106}      % Dark slate blue for borders
\definecolor{bgcolor}{RGB}{245,245,250}         % Light lavender background

\begin{document}

% Use a sans-serif font similar to the one in the image (requires XeLaTeX)
% TeX Gyre Heros is a free alternative to Helvetica
\setsansfont{TeX Gyre Heros}
\sffamily

\begin{tikzpicture}[
    % Set global options for the diagram
    node distance=1.5cm and 2.5cm, % Vertical and horizontal spacing between nodes
    % Define a style for the background color
    page/.style={
        background rectangle/.style={fill=bgcolor}, 
        show background rectangle
    },
    % Define a base style for all blocks
    block/.style={
        rectangle, 
        rounded corners=8pt, 
        draw=bordercolor, 
        very thick,
        text=black,
        font=\sffamily\huge,
        minimum width=4cm,
        minimum height=1.2cm,
        text centered
    },
    % Define specific styles for each type of block using color gradients
    convstyle/.style={
        block, 
        left color=colorConvLeft, 
        right color=colorConvRight
    },
    bottlestyle/.style={
        block, 
        left color=colorBottleLeft, 
        right color=colorBottleRight
    },
    concatstyle/.style={
        block, 
        left color=colorConcatLeft, 
        right color=colorConcatRight
    },
    % Define the style for arrows
    arrow/.style={
        -Stealth, % Arrowhead type from arrows.meta library
        draw=black,
        very thick
    },
    % Style for annotations
    label/.style={
        font=\sffamily\huge
    }
]

% Apply the background color to the entire picture
\begin{scope}[page]

% --- Node Placement ---
% Place nodes relative to each other for a flexible layout
\node[convstyle]                               (conv1)   {Conv};
\node[bottlestyle, below=of conv1]             (bottle1) {Bottleneck};

% Position the left 'Conv' node to be vertically between the top two nodes
\coordinate (midpoint) at ($(conv1.south)!2.!(bottle1.north)$);
\node[convstyle, left=of midpoint]             (conv3)   {Conv};

\node[bottlestyle, below=of bottle1]           (bottle2) {Bottleneck};
\node[concatstyle, below=of bottle2]           (concat)  {Concat};
\node[convstyle, below=of concat]              (conv2)   {Conv};

% Place annotation text nodes
\node[label, right=0.3cm of conv1] (c3k) {C3k};

% --- Arrow Drawing ---
% Draw the main vertical connections
\draw[arrow] (conv1.south)   -- (bottle1.north);
\draw[arrow] (bottle1.south) -- (bottle2.north);
\draw[arrow] (bottle2.south) -- (concat.north);
\draw[arrow] (concat.south)  -- (conv2.north);

% Draw the forked path
% 1. Define a 'fork_point' on the path between conv1 and bottle1
\path (conv1.south) -- (bottle1.north) coordinate[pos=0.25] (fork_point);
% 2. Draw a right-angled arrow from the fork point to the left 'Conv' node
\draw[arrow] (fork_point) -| (conv3.north);
% 3. Draw a right-angled arrow from the left 'Conv' node to the 'Concat' node
\draw[arrow] (conv3.south) |- (concat.west);

% --- Decorations ---
% Draw the brace for the 'N' annotation
\draw[decorate, decoration={brace, amplitude=10pt}, very thick, draw=black] 
    ($(bottle1.east)+(0.3cm,0)$) -- ($(bottle2.east)+(0.3cm,0)$) 
    node[midway, right=10pt, label] {N};

\end{scope}
\end{tikzpicture}

\end{document}